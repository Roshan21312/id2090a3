\section{ME22B106}
\begin{enumerate}
    \item Name: Anuj Sreenivasan
    \item GitHub User ID: AnujSreenivasan
\end{enumerate}

\textbf{\large{Hooke's Law :}}
\begin{equation}
    F_{x}(x) = -kx
\label{eqn1}
\end{equation}
The above equation, known as Hooke's law, is an empirical law stated by 
the $17^{th}$ century physicist Robert Hooke\footnote{Hooke's law,
Wikipedia, Last updated on 22 November 2022}. It states that the force 
exerted by a spring [$F_{x}(x)$] is directly proportional to the extension 
of the spring from its natural length [x]. Note that equation \ref{eqn1} 
confines the spring to a singular axis for simplicity. k is a positive 
value called the Force constant of the spring and is a characteristic 
property of a given spring. Since k is positive, the equilibrium at x=0 is 
stable. As seen in the equation, the force applied is in the opposite 
direction of the extension/compression of the spring. Thus, this force 
gives rise to oscillations\footnote{Classical Mechanics by John R. Taylor, 
Chapter 5, pages 161-162 under section 5.1 (Hooke's Law)}.  
