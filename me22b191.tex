\section{ME22B191}
Name : Soham Shah \\
Github id : sohamshah557@gmail.com 
\subsection{L'Hôpital's Rule(The Definition)}
\underline{\large{\textbf{Theorem}}(L'Hôpital's Rule)}: Let $f(x)$ and $g(x)$ be differentiable on interval \textit{I} containing $a$, and that \(g'(a)\neq\ 0\text{ on } I \text { for } x \neq\ a\). Suppose that \\
\[
    \lim_{x\to a}\frac{f(x)}{g(x)}=\frac{0}{0}\hspace{1cm} \text{or}\hspace{1cm} \lim_{x\to\infty}\frac{f(x)}{g(x)}=\frac{\infty}{\infty}
\]
Then as long as the limit exists, we have that \\
\[
\lim_{x \to a}\frac{f(x)}{g(x)} = \lim_{x \to a}\frac{f'(x)}{g'(x)}\footnote{\url{https://www.math.arizona.edu/~tlazarus/files/L\%27Hopital.pdf}}
\]
There is an analogous version for when \(a\) is \(\infty\)
or \(-\infty\).What this theorem essentially says is
that if you tried to compute the limit of a ratio of functions, but you get the indeterminate forms \(\displaystyle\frac{0}{0}\) or \(\displaystyle\frac{\infty}{\infty}\) , then you can compute the limit of the ratio of the derivatives of those functions instead.
However, take caution that it is not necessarily a short cut. When encountering limits that we have
seen before, it may be faster to use a different technique than L'Hôpital's Rule Rule. Also note that
we are \textbf{\textit{not}} taking a quotient rule. We just take the derivatives of the top and the bottom of the
fraction and leave them there.


\subsection{Examples}
L'Hôpital's Rule is a powerful tool for finding limits, but it should be used with caution, as it can sometimes give incorrect results. It is always advisable to check the limit using other methods when possible, and to use L'Hôpital's Rule as a last resort.

  
