\title{\textbf{ASSIGNMENT - 3}}
\author{Rishi Thalakoti \\ME22B186}

\begin{itemize}
    \item NAME : Rishi Thalakoti
    \item ROLL-NO : ME22B186
    \item GITHUB USER ID : RishiThalakoti
\end{itemize}
\textbf{\LARGE Föppl–von Kármán equations\footnote{reference from \url{https://en.wikipedia.org/wiki/Föppl–von_Kármán_equations}}}
\section*{Introduction}
The Föppl–von Kármán equations, named after August Föppl and Theodore von Kármán, are a set of nonlinear partial differential equations describing the large deflections of thin flat plates. With applications ranging from the design of submarine hulls to the mechanical properties of cell wall, the equations are notoriously difficult to solve, and take the following form:
\begin{equation}
 \frac{\textit{E}\textit{h}^3}{12(1-\nu^2)}\nabla^4w - \textit{h}\frac{\partial}{\partial x_\beta}\left(\sigma_{\alpha\beta}\frac{\partial w}{\partial x_{\alpha}}\right) = P  
\end{equation}

\begin{equation}
\frac{\partial\sigma_{\alpha\beta}}{\partial x_\beta} = 0
\end{equation}
where :
\begin{itemize}
    \item\textit{E} is the Young's modulus of the plate material (assumed homogeneous and isotropic),
    \item$\nu$ is the Poisson's ratio
    \item\textit{h} is the thickness of the plate
    \item\textit{w} is the out–of–plane deflection of the plate
    \item\textit{P} is the external normal force per unit area of the plate
    \item$\sigma_{\alpha\beta}$  is the Cauchy stress tensor
    \item $\alpha, \beta$ are indices that take values of 1 and 2 (the two orthogonal in-plane directions)
\end{itemize}
The 2-dimensional biharmonic operator is defined as

$$\nabla^4 \textit{w} := \frac{\partial^2}{\partial x_\alpha \partial x_\alpha}\left[\frac{\partial^2 \textit{w}} {\partial x_\beta \partial x_\beta }\right] = \frac{\partial^4 \textit{w}}{\partial x_1^4} + \frac{\partial^4 \textit{w}}{\partial x_2^4} + 2\frac{\partial^4 \textit{w}}{\partial x_1^2 \partial x_2^2}$$
Equation (1) above can be derived from kinematic assumptions and the constitutive relations for the plate.\\
Equations (2) are the two equations for the conservation of linear momentum in two dimensions where it is assumed that the out–of–plane stresses ($\sigma_{33},\sigma_{13},\sigma_{23}$) are zero.
\section*{Pure Bending\footnote{reference from \url{https://en.wikipedia.org/wiki/Pure_bending}}}
For the pure bending of thin plates the equation of equilibrium is
\begin{equation}
    \textit{D}\Delta^2 \textit{w} = \textit{P} ,\tag*{where}
\end{equation}
$$\textit{D}:=  \frac{\textit{E}\textit{h}^3}{12(1-\nu^2)}$$
is called flexural or \textit{cylindrical rigidity} of the plate.
