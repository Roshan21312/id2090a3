\section*{\underline{Equation}}
\[ pH=pK\textsubscript{a}+log\frac{[A\textsuperscript{-}]}{[HA]}\]
Where [A\textsuperscript{–}] denotes the molar concentration of the conjugate base of the acid and [HA] denotes the molar concentration of the weak acid.
\section*{\underline{Explanation}}
The Henderson-Hasselbalch equation provides a relationship between the pH of acids (in aqueous solutions) and their pKa (acid dissociation constant). The pH of a buffer solution can be estimated with the help of this equation when the concentration of the acid and its conjugate base, or the base and the corresponding conjugate acid, are known.
The Henderson-Hasselbalch equation can be written as:
\[ pH=pK\textsubscript{a}+log\frac{[Conjugate  base]}{[Acid]}\]
Where [A\textsuperscript{–}] denotes the molar concentration of the conjugate base of the acid and [HA] denotes the molar concentration of the weak acid.An equation that could calculate the pH value of a given buffer solution was first derived by the American chemist Lawrence Joseph Henderson. This equation was then re-expressed in logarithmic terms by the Danish chemist Karl Albert Hasselbalch. 

\subsection*{\underline{Limitations of the Henderson-Hasselbalch Equation}}
The Henderson-Hasselbalch equation fails to predict accurate values for the strong acids and strong bases because it assumes that the concentration of the acid and its conjugate base at chemical equilibrium will remain the same as the formal concentration (the binding of protons to the base is neglected).
\section*{}
Name: V Madhumitha\\
Github user-id: Madhumitha2305\\
Reference:\footnote{https://chem.libretexts.org/Ancillary\_Materials/Reference/Organic\_Chemistry\_Glossary/Henderson-Hasselbach\_Equation}
