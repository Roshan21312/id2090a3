

\section{Introduction}
Iam Bandarupalli VenkataSiva Prathik , github id: Prathik747 , the author of this file , will breif about the formula which newton gve for speed of sound in air and 
how laplace identified the error in newtons assumption and corrected it to give accurat results..
Newton proposed that the propogation of sound through a medium is isothermal process and he gave the formula for speed of sound in air as 
\begin{equation}
    \text{speed of sound in air} =\sqrt{P/d}
\end{equation}
\section{Discussion}
where P represents Pressure of gas in air , d represents density of gas in air. Later Laplace demonstrated that adiabatic conditions are required for the propagation of sound waves. The fact that compression and rarefaction in the air will happen quickly indicates that an adiabatic situation exists when the change in applied heat is zero. This is because the thermal conductivity of air is so low. Heat won’t move into or out of the system as a result. This is referred to as Laplace correction for sound waves in an atmosphere or a gaseous medium. The Laplace Correction is used to modify the gas’s sound speed. Laplace created a theoretical change as well as one that is application-specific. As a result, Newton’s Formula is sometimes known as a Laplace Adjustment.
So he revised the formula for speed of sound in air as :
\begin{equation}
    \text{speed of sound}=\sqrt {($\gamma*P/d)} 
    
\end{equation}
where $\gamma$ represents the ratio of Cp and Cv for the gas , and as earlier P represents pressure of the gas and d represents density of the gas..

\footnote{https://www.geeksforgeeks.org/laplace-correction/}
