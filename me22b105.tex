\section{me22b105}

\[ s_1-s_0 = ut + \frac{1}{2}at^2 \]

where:
\begin{itemize}
    \item $s_0$ is the particle's initial position
    \item $s_1$ is the particle's final position
    \item u is the particle's initial velocity
    \item a is the particle's acceleration
    \item t is the time interval of the particle's motion
\end{itemize}

In physics, equations of motion\footnote{Referenced from - Understanding Physics Mechanics Volume-1 By DC Pandey, Section 6.6, Page 136} are equations that describe the behavior of a physical system in terms of its motion as a function of time. More specifically, the equations of motion describe the behavior of a physical system as a set of mathematical functions in terms of dynamic variables. These variables are usually spatial coordinates and time, but may include momentum components. The most general choice are generalized coordinates which can be any convenient variables characteristic of the physical system. The functions are defined in a Euclidean space in classical mechanics, but are replaced by curved spaces in relativity. If the dynamics of a system is known, the equations are the solutions for the differential equations describing the motion of the dynamics.




\begin{flushleft}
[
Name - Anish Rayaguru
GitHub ID- anishrayaguru26
]

\end{flushleft}
